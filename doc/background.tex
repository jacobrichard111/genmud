\chapter{Background Information}

This chapter will present some of the background material for how the
game works. It also includes some of the internal editing information
you can use. ``Things'' will be discussed in more depth in the next
chapter.

\section{Network}

The core of the MUD is the network code. On bootup, the database is
read in using the function load\_server(), and a listening socket is
set up. For a good good book on this, get W. Richard Stevens'
{\it{Unix Network Programming Vol 1., Sockets and XTI}} This code is
contained in init\_socket(). This is a standard sequence of commands,
socket(), setting SO\_REUSEADDR, bind (), setting NONBLOCK, listen
(). Usually people have these commands all separated and have an error
message for each (look at apache or quake or something), but I figured
since bind() is the thing that doesn't work if something fails, just
make it one thing.

Once the listening socket is set up, the game enters an infinite
loop. Ten times a second, the following things happen.

\begin{enumerate}

\item  We get the current time of day.

\item The FD\_SET input\_set has all of its bits zeroed. Note that I
only use one FD\_SET. It isn't strictly necessary to use some
output\_set and some exception\_set, but you can. My code is just not
very robust, since it started out as a simple chat room. Set the
listen\_socket number (usually 0) to listen for inputs (new
connections).

\item Then we go down the list of file\_descriptors (connections), and
we set the bits corresponding to what fd's the data chunks are hooked
up to in the input\_set. (i.e. we tell select () that there might be
input coming in through these connections). At the same time, check
for idling.

\item Then run select() which polls all of the descriptors for input.

\item We then add a new connection if needed. 

\item Then we go down the list of all file descriptors and read input
from each (if necessary). The inputs are then immediately parsed for
aliases and multiple commands. The alias code takes place in
read\_in\_command () and will be discussed below.

\item Then, we do whatever command was read in during
read\_in\_command (if any).

\item Then, every pulse, events get updated. Each thing has a few
events on it that repeatedly get updated at different intervals to
give the illusion of action and a dynamic world. These things are
explained more in the chapter on events.

\item Then, we go down the event list (for scripts) and execute any of
those that are pending.

\item Then, finally we flush the output buffers for each connection
and find the time at the end of the cycle, and then put the process to
sleep until it is needed again. Note that select (NULL, NULL, NULL,
\&waiting\_time); is essentially the same as usleep().

\end{enumerate} 

\section{Data}

The database has many components. The most important components are clearly the things. However, there are other types of data which must be read in. These can essentially be found in the ../wld directory in the .dat files.

A quick note about how the data is set up. I use a simple 

Keyword Data

format so that the text files are easily read and easy to change by hand, or
 by some shell program if necessary.

\begin{enumerate}

\item Alignments: This the data on the various alignments within the
game. There is OLC for this (align edit $<$number/name$>$, and this is
automatically saved).

\item Areas: This is a list of the ``area'' files which will be read
in. You really need start.are or something like that...the area with
vnum 1, so that the core things needed for the world can be read in.

\item Channels: This reads in all the information about channels like
chat, say, tell, yell etc... There are several flags in serv.h:
CHAN\_XXXXX, but the editing for this is only done in the channels.dat
file. You can use those as a template for making other different
channels if needed.

\item Clans: There is interesting clan code here. There are several
types of clans like clans, sects, schools, academies or whatever you
want. It is set up so each person can be a member of one of each type
of clan. It is also set up to make it easy to add new types of clans,
although you do have to do coding. This file saves the data on each
clan that actually exists, from rosters to items stored.

\item Code/Triggers: Scripts have three main components, code that is
executed, triggers which are attached to a thing and on a certain
action cause code to be executed, and events which are created when a
trigger is triggered. These files store the interpreted code and the
triggers which cause the scripts to happen. There will be much more on
scripts later.

\item Damage: dam.dat just holds a list of all the damage
messages. Note how it uses only one message per damage level. This is
due to the really good markup function act() which is something I
always hated about my old code because of all the awful void* pointers
in it.

\item Helpfiles: help.dat contains the helpfiles accessable within the
game. It hashes the keywords and has several keywords pointing to the
same helpfile text. This should work if you help reload\_helps...and
it loads more helps in. The helpfiles are in a standard format: The
helpfiles start with the keyword HELP followed by a number or word.
The words are MORT BLD MAX. MORT sets the helpfile to level 0,
BLD sets the helpfile to BLD\_LEVEL, and MAX sets the helpfile
to MAX\_LEVEL. Helpfiles have a list of keywords that can be used
to access the helpfile, then admin text that only MAX\_LEVEL people
can see (so you can have a single helpfile and have player info
and the special admin info in it), regular text that anyone can
see, and a list of See Also words that appear at the bottom of
the helpfile. 

\item Notes: The notes just get read in as a big long list. Note that
admins can append responses to notes if they choose.

\item Playerbase: A small piece of the player's data from level, to
last login to email is stored online so people can use the finger
command. Admin fingering gives more information.

\item Pkdata: pkdata.dat stores the global pk info top lists. This is
used for the pkdata command so that people can brag and make fun of
each other.

\item Races: The races are stored in races.dat and are essentially the
same as alignments.

\item Score: The score file is an interpretted file which is read in
from here. It uses the string parsing and variable replacement used
for scripts and prompts. Those things will be discussed later.

\item Sitebans: This is a list of all sites that have been banned.  I
think it works, but I haven't checked it in forever and I forget.

\item Socials: These are dippy little commands you can type in that
just let you send a message to everyone in the room. There are only 2
strings along with the social name, because the act() function does
all the work.

\item Societies: These are collections of mobs that work together to
achieve common goals like building cities and raiding other societies
and gathering resources.

\item Raids: These are used to keep track of societies that are
attacking other societies.

\item Rumors: When societies do things, their actions are recorded as
rumors that can then be accessed by players by typing news at a
mob. When they type news, they may hear about recent events, or find
out what ally society needs help either in terms of combat assistance,
or raw materials needed. Admins can access all current rumors by
typing rumor.

\item Spells: Spells are stored in a database. The actual
spells/poisons/traps actually use the spell data to be
``cast''. Skills are generally hardcoded, but the skill level and
guilds needed etc...are stored here. There will be more on this later.

\item Worldinfo: This is a snapshot of the current game database. That
way, when the game crashes or reboots, the status hasn't really
changed...mobs even keep special flags and are in the same locations,
and they even remember who attacked them last.

\item Weather/Time Info: Current weather and game time.

\item Things: These are read in from area files and will be dealt with later.

\end{enumerate}

        
\section{File I/O}

The way files are used is pretty simple. Basically, I use the old
Keyword Data method, so the loops that read stuff in from files search
for keywords, then if they find the keyword, they read in the data
after it. The functions which read/write stuff from/to files can be
found in fileio.c The read functions basically eat up whitespace, then
follow with an appropriate fscanf. {\i{This is now a series of
getc()'s since fscanf was kind of slow when I tried loading up a
million things from worldinfo.dat.}}    

\section{Arguments}

When you type in commands within the game, many times you need to use
several arguments. The code lets you enter multiple words as a single
argument by enclosing them in single quotes. So, for example, c 'magic
missile' whitie would cast a magic missile spell at a whitie.

\section{Multiple Commands At Once}

There is a special character used within the text data files to
represent the end of a string. If you are familiar with Diku, you know
that the tilde $\tilde{}$ is this character. I don't use the tilde
because someday I might make my script engine suck less, and if I do,
I will need \&, $\vert$, $\tilde{}$ for bitwise operators. So, for
that reason, I use the backwards quote...accent? Which is `. On my
keyboard it is the character below the tilde $\tilde{}$.

Now, how does this let you enter multiple commands at once? Since we
can't allow you to enter a ` into a text stream, we need to do
something with it. What happens is that in your input buffer, if it
reaches a `, it assumes that is a $\backslash$n$\backslash$r. That
means you can enter a string of commands all in a row like n`n`e`e`s
and then the game will execute n n e e s as soon as possible (one
command per pulse). This can be useful if you are trying to run past a
dangerous part of a dungeon, or trying to execute a series of commands
quickly for pkill and you don't feel like making an alias...like you
know the enemy is n e from here so you type in n`e`kick enemy and you
would be able to attack them quickly.

\section{Helpfiles}

The helpfiles within the game are accessable by using the ``help
$<$keyword$>$'' command...and note that $<$keyword$>$ can have several
words surrounded by single quotes. Most of the skills and spells have
helpfiles within the game, and many of the other features have
helpfiles which are appropriate for players. This document will have
some more in-depth information.

\section{Aliases}

There are aliases within the game. They have several nice
features. First, you can have arguments which get replaced when the
aliases are typed in (3 arguments only but it's easy to up that if you
choose.) Also, you can make aliases with multiple commands in a single
alias, which is helpful also.

To have variable arguments put into an alias, use the strings @1 @2 @3
to represent inserting arguments 1, 2 and 3. Don't worry if you have
an alias with arguments and you don't enter enough arguments...the
game will just leave them blank. This is useful as I will show you
with a couple of examples below.

If you want to have multiple commands within a single alias, you can
do this, too. The way to do this is to delimit the ends of commands
with a *. So, if you wanted an alias that made you go n n e e s, and
you wanted it to be triggered when you type in zz, you type

alias zz n*n*e*e*s

There are limits on the length of aliases, and on how long the strings
can be once they are replaced, but I am not sure if I caught all the
cases with this, so possibly there could be alias expansions which end
up too long. Dunno tho. 

Now, getting back to arguments in aliases, I have several building
aliases I use a lot. The directions in the game are n, s, e, w, u,
d. There are commands where you can type $<$dir$>$ dig
$<$blank/num$>$, and that will link/create rooms in that
direction. Using east as an example, I have aliases in all directions
which are like this ee, dd, ss, nn, uu, ww, where ee is set up by

alias ee e dig @1

What this means is the following. If I am in a room, and I type ee,
the game finds the next available room in the area and creates it and
links the rooms. If I use an argument, then if the room associated
with that number exists (and is in the same area), it links the room
you are in to that room. If the room doesn't exist, but it could exist
in the same area, it creates the other room and links the rooms.

Also, I have aliases like the following, for all of the dirs n, s, e,
w...I have aliases set up to make simple doors. I have aliases called
nd, sd, ed, wd where ed is defined to be

alias ed v e v1 3*v e v2 3*v e word door

This is used to put data into the east\_door ``value'' in the thing I
am editing. The alias sets up door and closed door flags and resets
them, and gives the exit the name ``door''.  There will be more on
values later, but they are things that can be exits, money, weapons,
teachers, and just about any other piece of info you can imagine. They
make the code a lot easier to deal with. I suggest you use them
instead of adding new data or pointers onto things (except for pc's
where I ended up doing everyhing by hand for some reason.. :P).

\section{String Editing}

Since the game is played in telnet line mode, we don't have a
character editor. We do have a string editor, however, and it lets you
some of the simple things you would expect.

If you are editing something which I decided was a long string and might
require a longer string than just a single line, you get put into the
string editing mode. Generally when you enter text, it just gets
appended to the string. There are several places where the string
editor is used. Thing descriptions, notes, writing script code are
just a few. There are also several editing/formatting commands
available in the string editor. They all have the form .$<$letter$>$,
and must be typed at the very start of a line.

\begin{enumerate}

\item .d This ends your string editing.

\item .c Without an argument, this clears the entire string. With an
argument, it clears a single line number. So, .c 10 would clear the
line after the number 10 on the left.

\item .r $<$old$>$ $<$new$>$ substitutes text. This also accepts
arguments with several words in them if you encase them in single
quotes. This is useful say if you need to change a single instance of
the: ...going to the new place..., and you wanted ``the'' changed to
``some'', you could type

.r 'g to the ne' 'g to some ne'

and that will only change that one instance of the. It isn't as pretty as a query/replace or anything, but it works. 

\item .i $<$line/blank$>$ $<$string$>$ inserts a line at line
$<$line$>$. If the first word after the .i is not a number, it assumes
you want to put the string at the very start of the whole string, and
it moves everything else down. If you replace line 10, for instance,
then the current line 10 gets moved to line 11, and your current input
line gets put in as line 10.

\item .f This formats the text. Note that this kills off any
paragraphs, extra spacing you might have in the string. You cannot
format ``code'' for the scripts, because that relies on newlines to
determine where the end of a command line is. This also makes it so
any color codes in the text are ignored when we format, so putting a
multicolored word in there doesn't make that line only have 10 letters
in it, while a line without any colors would have around 75-80.


\end{enumerate}

\section{Hexadecimal numbers}

Hexadecimal numbers or hex numbers for short are a valuable tool for
representing single bits as numbers. Computers are systems that work
on binary arithmetic: arithmetic in terms of 0's and 1's. It isn't
always convenient to look at a string of numbers in terms of 0's and
1's, so an extension of the regular decimal number system was developed.
Basically, the decimal numbering system has 10 symbols: 0,1,2,3,4,5,6,7,8,9.
If you want to represent something larger than 9, you use two or more
symbols instead of just one. Hexadecimal numbers work the same way,
except you now have 16 symbols: 0,1,2,3,4,5,6,7,8,9,A,B,C,D,E,F.
The symbols 0-9 are the same as before, but A represents ``10'',
B represents ``11'', C represents ``12'', D represents ``13'', 
E represents ``14'', and F represents ``15''. Just as the 
``base 10'' numbering system has symbols for all numbers up to
1 below 10, the hexadecimal numbering system has symbols for all
numbers up to 1 below its base. So, since there are symbols for
all values up to 15, the hexadecimal system is a ``base 16''
numbering system.

This is important because as  I said before, computers the binary
numbering system consisting of 0's and 1's. Hexadecimal numbers give
us a bridge between our well-understood decimal numbering system and
the hard-to-use binary numbering system. 

The relationship between hexadecimal numbers and binary numbers is
quite simple. Imagine a binary number with 4 symbols. Each symbol is
either 0 or 1. An example of such a number is 0101. Note that this
number has 4 spaces for symbols, and there are two choices for
each symbol. The total number of 4 digit binary numbers is
therefore $2\times 2\times 2\times 2$ because for each of the
four slots there are two choices, so to get the total number of
choices for all four slots at once, we multiply the number of
choices we have at each step. There's that number 16 again. 

Basically what hexadecimal numbers do is represent binary numbers
where each hexadecimal digit represents a block of 4 binary digits (or bits).
When counting in decimal numbers, you know that 

$$51274 = 5\times 10^{4} + 1\times 10^{3} + 2\times 10^{2} + 7\times 10^{1} + 4\times 10^{0}.$$

Binary numbers work in the same way. When you write a string of binary
digits out to represent a larger number, the digits represent a sum
of coefficients times a power of 2, instead of 10. So, the binary number


$$11010 = 1 \times 2^{4} + 1 \times 2^{3} + 0 \times 2^{2} + 1\times 2^{1} + 0\times 2^{0} = 26. \hbox{ base 10}$$

So, given any block of 4 bits, the smallest number is 0000, which is
equal to 0 base 10. The largest number is 1111 which is equal to 8 + 4
+ 2 + 1 = 15 base 10. And, given any other possible combination of a
string of four 0's and 1's, this string can be converted into a
(unique) decimal number with a value between 0 and 15. That's how the
map works. The string of 4 bits has a decimal value between 0 and 15,
and the hexadecimal digits represent all numbers from 0 to 15, so you
convert the string of bits with a certain value to the hexadecimal
digit with that same value.


This map goes the other way, too. So, the hex number B is equal to 11,
which is equal to $8 + 2+ 1$ which is equal to 1011.

If you have longer strings of bits, you just break them up into groups
of 4 bits and then apply the same relationship shown above to each of
those sets of four bits.

For example, the number 010110101010110 in binary can be split into
groups of four bits (STARTING FROM THE RIGHT!) like this: 010 1101
0101 0110. The leftmost set of bits only has 3 numbers, so just assume
that the top digit is 0. The decimal values for each of these
sets of bits are 2 13 5 6, and so the conversion of these decimal
values to hexadecimal values becomes 2 D 5 6 or 2D56. 

Also, note that if you ever use a hex editor to look for numbers in a
binary file on a computer, the bits of the numbers are 
stored out of order because of how the numbers are processed
inside the computer.


These hex numbers are used a lot in serv.h to represent bit for
flags. There are several ways to represent a binary number with all
0's and a single 1 at one point. Inside the machines I currently use,
int's are 32 bits. So, it takes 32/4 = 8 hexadecimal digits to 
represent that number, since each hexadecimal digit represents 4
bits.

Hexadecmial numbers that the compiler will read in from source
files start with the symbol 0x followed by several hexadecimal
digits. There can be very few digits, such as a number like
0x4A or many like 0xFA49C13B. 

Numbers with single nonzero bits can be represented in several ways.
One way is to use the left shift operator $<<$, by writing things like
 $BIT\_0 = (1 << 0)$, or $BIT\_14 = (1 << 14)$. These shifts work up
to and including the left shift by 31 places (which is a single 1 in the
leftmost spot of a string of 32 bits). When people make up their lists
of flags, they sometimes use the BIT\_XX notation or the $1<<X$ notation,
but I prefer to use the hex notation like the following.

If a hexadecimal number has all 0's in it except one digit is a 1, 2,
4 or 8, then that hexadecimal number represents a binary number with a
single nonzero digit. You can see this by remembering that each digit
of the hex number represents a string of four bits. All of the strings
of bits represented by the 0's are all strings of four 0's. The one
nonzero hexadecimal number is an exact power of 2, and so therefore
represents a single bit: (1 = 0001, 2 = 0010, 4 = 0100, 8 = 1000).


So, in serv,h, there are lists of flags and these long numbers after them like:

\begin{verbatim}
#define OBJ_GLOW         0x00000001  /* Hit ghost */
#define OBJ_BLESSED      0x00000002  /* Hit demons */
#define OBJ_HUM          0x00000004  /* Hit undead */
#define OBJ_POWER        0x00000008  /* Hit dragons */
#define OBJ_CURSED       0x00000010  /* Hit angels */
#define OBJ_WATER        0x00000020  /* Can hit fire mobs */
#define OBJ_FIRE         0x00000040  /* Can hit water mobs */
#define OBJ_AIR          0x00000080  /* Can hit earth mobs */
#define OBJ_EARTH        0x00000100  /* Can hit air mobs */
...
\end{verbatim}

Notice how the hex numbers are set up? For each hex digit, there are
four flags used, the 1, 2, 4, and 8 bits. Then, once those 4 bits represented
by that hex digit are used up, the next digit is used up by using 
numbers to represent each of the bits being set. This can continue
until we reach the top number 0x80000000, but after that you will
need to make a new type of flag if you need more flags. This is not 
as versatile as using a real bitvector, but it works pretty will
since it's easy to add new kinds of flags.

\section{Octal Numbers}

Just for the sake of completeness, octal numbers are base 8 numbers using
the digits 0-7 to represent strings of 3 bits with a single
symbol. I didn't use them in here much if at all, but they are out
there if you need them.

\section{Colors}

The color code in the game is unified. It works in the following
way. Since ANSI colors come in the form $\backslash$x1b[(0 or 1);3(0
to 7)m I decided to be lazy and since there are only 16 colors I use
(since I don't want to have flashing/inverse since they are annoying),
I decided to use the hexidecimal digits to represent colors. So, a
color code looks like \$ followed by a hexidecimal digit. Within the
color code, we have $\backslash$x1b[N;3Mm, where N is 0 or 1
(regular/bright), and M is 0 to 7: black, red, green, yellow, blue,
purple, cyan, white. (And incidentally there is even more of a pattern
here using rgb and bright/not bright but I won't get into it. So, we
use the hexidecimal letters 0-F to represent the 16 different colors
possible. Convert the hexidecimal digit to a number num, and then we
add the following string to our text: $\backslash$x1b[(num/8);3(num
\%8)m. So this is a list of the colors and their corresponding codes:

\begin{enumerate}

\item \$0 black

\item \$1 dark red

\item \$2 dark green

\item \$3 dark yellow/brown

\item \$4 dark blue

\item \$5 dark purple

\item \$6 dark cyan

\item \$7 gray (dark white)

\item \$8 dark gray (bright black)

\item \$9 bright red

\item \$A bright green

\item \$B (bright) yellow

\item \$C bright blue

\item \$D bright purple (magenta)

\item \$E bright cyan (sky blue)

\item \$F white (bright gray)

\end{enumerate}

So, afaik you can add these colors to any strings you want, names,
short descs, long descs, descs, notes, spell messages, act() strings,
etc.. Be warned, though, sometimes the strings in question really
shouldn't have colors so in a few cases I think it's disabled... like
with thing names and stuff like that. Also, be warned if you do put
those into something like thing names, you won't be able to find those
things because their names will be all screwed up.

\section{Messages}

There are a few ways in which messages can be sent to characters
within the games. The most basic is the void stt (char *txt, THING*
target); function (send\_to\_thing :P), which basically checks if the
thing has a network connection, and if so, it appends the string to
the file descriptor's buffer.


You can send a broadcast message to everyone by using the function echo (char *message);

There is a nice simple idea for a markup language I got from the old
codebase. If you have ever seen act() in Diku, you know it sends a
message to some or all people in a room, and replaces certain text
keys with names of things. This is the same idea, but it's written the
way it should have been done IMO. No awful void* pointers and junk
like that. Btw, this function was one of my biggest pet peeves about
how the old code worked.

The prototype is as follows:

void act (char *msg, THING *th, THING *use, THING *vict, char *txt, int type);

The message is the string that gets parsed and sent to some or all of the things in a room.

th is the thing which is doing the acting and all other messages are relative to it.

use and vict are other things being mentioned in the string. 

Type tells what kind of a message you send, is it TO\_CHAR (only to
th), or TO\_VICT (only to vict), TO\_USE (only to the use), TO\_ROOM
(to all except th), TO\_NOTVICT ( all except th and vict), or TO\_ALL
(to all in room). This last one is the most fun, since it means you
don't need three stupid messages for every act(). To see how this
might be useful, look at socials. It lets me be lazier and not have to
type so much junk because the code does more work for me.

You can see from the code the kinds of places this is used. Generally,
@$<$digit$>$$<$letter$>$ is used to represent a name or a pronoun
associated with a thing. The $<$digit$>$ can be 1, 2, or 3,
representing th, use, or vict respectively. The $<$letter$>$ can be n
(name (short desc)), p (possessive name), h (he/she/it), etc...there
is more help on this in the helpfiles.

The char *txt gets put into the text anywhere there is a @t. 

There is another interesting feature and is why you can use one
message for everyone. There is a replacement @s which is essentially
used for verb conjugation relative to who is acting, and who is seeing
the act. For example. Lets say the name of th is Bob, and the name of
vict is Fred.

act(``@1n @t@s @3n very hard., th, NULL, vict, ``slash'', TO\_ALL);

If we wanted to see what the code does to this string, let's look at it.

We send the message to everyone, so if we are sending it to th,

Bob sees You slash Fred very hard.


Fred sees Bob slashes you very hard.

Everyone else sees Bob slashes Fred very hard.

It does not work with all the funky English verbs out there, but it
seems to do ok for the cases I have thrown at it.


