\chapter{Combat}

\section{Introduction}

The combat system is pretty simple. I realize that some people like
combat systems where you have to pick from 100 moves and defensive
positions and you're constantly trying to guess what your opponent
will do and each move has lagtime and all that blah.... but I don't
like that. That seems too much like a twitch game like quake where
combat is heavily dependant on the abilities of the PLAYER as opposed
to the CHARACTER. I feel that is the wrong way to run an RPG because
you're supposed to be building up a character, not learning how to
shoot 1000 shots a second in Quake. (I'm not saying Quake is bad, but
it's a different type of game.) 

All of the functions mentioned below should be in fight.c but if not,
you know how to use grep to find them.

I want combat to have several characteristics that I like. First, you
should not die too quickly. You should be able to predict somewhat how
your health will drain once you have experience playing the
game. However, there is some randomness (such as critical hits) that
can lead to extreme damage over a short period of time.

I don't want combat to have too many different options. I never really
got the point of MUDs where they have 10 kinds of kick and 7 kinds of
punch which are essentially kicks anyway.

One thing I do like is having combat skills that can be improved
incrementally by a person practicing more skills. This is appealing to
be because then everyone can have ``kick'', but only specialists will
get access to the advanced versions of kick. At the same time, the
advanced versions don't have different commands...they build on the
effectiveness of the original kick command. This plays into one of my
other philosophies that the CHARACTER should be the determining factor
in how combat goes, not the player.

\section{Generic Combat}

The core combat works as follows. Every 2 seconds or so, everyone
who's fighting tries to attack the person they're fighting. Each
person attacks at their own time, however since each fighter gets its
own combat update event. This process consists of doing some sanity
checks in update\_combat() followed by calling multi\_hit() for that
character. The multi\_hit() function goes down the list of weapons the
attacker is wielding and sees how many hits they will get.

There are several skills that affect
this like second attack, third attack haste, and multi wield. The
reason multi wield is not called dual wield is because it's possible
to have more than 2 arms. Essentially, you get your full complement of
attacks from your first weapon and half as many attacks from any
subsequent weapons you use.

 If the attacker is not a pc, it can have a
weapon value VAL\_WEAPON placed on it directly. That val weapon gives
the mob \_extra\_ attacks with a damage range given by the amounts
given when setting up the VAL\_WEAPON. If the attacker has no weapons,
the game checks to see if they can do some hand to hand hits. You
might want to add something in there for monks or martial arts, but
I'm not terribly interested in that.



Every time the attacker tries to hit the victim, first the game
checks if the attacker is using a weapon. The game calls one\_hit()
and the one\_hit() function does many things. Most combat hits are
normal, but some are special hits (like backstabs).

This function first checks if you hit, the formula is based on things
like your dex and level and stuff like that. Most of the time you hit.
This is something I picked up from the old Emlen code...you make the
hitroll really only affect the chance of hitting between 50 and 100
percent. 

If you hit, then you check if the weapon can hit the victim. This is
checked by checking the obj\_flags on the weapon vs the mob\_flags on
the victim. As stated in the previous section on flags, the 
power object flag allows weapons to hit dragon mobs and so forth.
This is a useful idea I think because with the way spells work,
you can have special spells that either remove mob defenses (like
dragonbane would remove a dragon flag on a mob...if you set it to
remove permanent flags) and another type of spell might put a
temporary (or permanent) power flag on a weapon that would allow it
to hit a dragon. This also leads to having a lot of nice spells
that alchemists can get by tiering them so that the temp spells
are low tiers, and the perma spells are more powerful.

Then, if the attack is not a special one, the game checks parry,
dodge, shield block, and blink using the check\_defense() function.
Note that in this function there are several tiers of ``parry''
and ``dodge'' skills which build on previous skills. So instead of
giving the player 10000 defensive options, the game knows that
you're able to defend yourself and it does it for you.

If the attack gets through the defenses then the damage is calculated.
I think most games use a ``dice'' system with 4d2 3d6 or whatever. I chose
to go with a range between a min and max number to give me more 
room to work with it. I knew that I would be allowing people to improve
their weapons (using the sharpen skill) so I didn't want to run out of
options for dice amounts. It does take the average of two random numbers
picked from this range so the distribution is triangular and centered
on the mean. 

Note that the ``basedam'' restricts the total damage done from a hit
to double the base weapon damage amount. That means no more loading 
up on damroll eq and wielding no weapon expecting to hit for
a lot of damage. Sorry.

There are some other additions like tinkers and concussion weapons
and enhanced damage and some stuff relating to skills like ravage
and backstab, and the warrior guild and implant ranks.

After that, the armor that the victim is wearing is used to
reduce the damage. Note that the armor reduces damage point
for point, so vs powerful hits, armor isn't so good.

After the armor damage is reduced, the damage resistance
given by sanctuary and body implants and total protection 
are removed. This damage reduction is on a percent basis
and is therefore much more powerful than armor for powerful
hits. 

Then after that, the damage() function is called.This function gets used a lot for things like drowning or rooms
trapping people or spells causing damage, so just be careful when
changing it because it really encompasses all types of damage a thing
can receive. The one\_hit function returns TRUE if the damage function
returns TRUE.

After the damage function is called, some code dealing with damaging
equipment and poisoned weapons follows.




\section{Damage Function}

This function handles all damage dealt out to players. There are
several things which are checked first.  Damage is doubled if the
attacker is permadeath, and halved if the victim is permadeath. Also,
AFF\_RAW on the victim doubles damage, outline increases damage based
on the level of the attacker, and AFF\_WEAK on the attacker halves the
damage done. The relics that the player alignments may have affect the
damage done also, along with built up society areas adding defensive
value. 

ALSO IF THE ATTACKER HAS LOST HIT POINTS THE AMOUNT OF DAMAGE THE
ATTACKER DOES IS ALSO DECREASED!!!!!

The next thing checked is powershields. These items store up
energy and use it to block damage from all types of
attacks. 

If the victim is casting or firing, then their concentration might be broken.


After all that is done, the messages are sent, and the hps of the victim
are reduced.

If the victim's hps go too low, the victim gets killed by calling the
get\_killed function that handles stuff like exp and corpses and stopping
fighting. 

Finally, there is some code that lets people autoflee in certain
cases.

\section{Fight Commands}

\bul Kill. You use the kill command to initiate combat with someone,
or to switch opponents. 

\bul Kick. The kick command lets you cause some extra damage against the
enemy you're fighting. This command requires skills to use
effectively. However, in keeping with my original premise of keeping
combat simple, there aren't 10 kinds of kick that you have to choose
from if you know several kick skills. The more advanced skills just
let you kick harder and more often. Also, note that if the kicker is a
PC and he has more than one pair of legs, he will kick more times. Be
careful with those ``centaur''-type races for this reason.


\bul Bash. The bash command lets you slam yourself into your enemy,
knocking him down to the ground for a few rounds. You will be unable
to use any combat commands for several rounds, including a few rounds
after he gets back up and starts fighting you again. This command is
useful if you want to cast a spell in combat, since spells require
precast incantations that can be interrupted by taking damage. So, if
you bash someone and immediately cast, you may get off a pspell of
decent power. There are several tiers of bash skills that increase the
length of time the victim is bashed, and also allow you to do some
damage while bashing.

\bul Tackle. Tackling someone puts the two of you into
``groundfighting'' position. That means you claw and attack each other
on the ground. When you're in this position, you are harder to hit
with normal melee hits, but in order to flee, you must first stand up
(which is hard to do) and then flee normally. You cannot tackle
someone if you have anything in your hands (including gems).

\bul Flee. Use this if you want to run away from combat. Setting your
wimpy will make you autoflee at a certain level of hps.

\bul Cast. You can cast offensive spells vs your enemies or sometimes
defensive or healing spells to help your friends. This stuff will be
explained more in the chapter on spells.

\bul Rescue. This lets you pull someone out of combat and make their
attacker start to attack you. If they are being hit by several
creatures, you will have to use this skill more than once. 

\bul Guard. This lets you guard another player, i.e. you will attempt
to autorescue that player periodically if they're being attacked.

\bul Load/Unload. These commands let you load or unload a ranged
weapon if you have the correct kind of ammo for it. The load command
with no arguments will try to load every ranged weapon you're wearing.

\bul Fire. The fire command is a delayed command that lets you attempt
to loose all of your ranged weapons on a victim. If you successfully
aim at your victim, you will unload all of your weapons into the
victim one by one. If the victim dies, you will unload the rest of
your shots from subsequent weapons into other creatures with the same
name as the killed victim. So, for example, firing at ``enemy'' will
keep you firing at all enemies until all of your shots for that round
are used up, or all opp align enemies are dead.

\bul Disengage. This command lets you step out of combat if you're not
being attacked. You may re-engage if you have autoassist on, so be
careful with this.

\bul Disarm. This lets you attempt to knock an item out of your
enemy's hand. It can be anything from a gem to a weapon to a potion to
whatever.

\bul Assist. This lets you start fighting on the side of whoever you
name (as long as it's appropriate).

\bul Backstab. This works a little differently here than it does
elsewhere. If you and your victim are not fighting, you hit for full
damage. If either of you is fighting, this is essentially a ``circle''
that does less damage. You cannot backstab someone attacking you. You
also cannot backstab a mob that's not fighting if you're fighting. But
again, there is no more ``circle'' command. My reasoning is this. In
pk, mages are constantly being rescued/disengaging/casting and so you
type bs sprite and it casts a spell, and now you can't backstab, so
you try to circle. Then, it gets rescued and you finish your circle
only to see that you can't circle anyone unless they're fighting. This
situation is annoying so I decided to code it out of existence. :)

\bul Flurry. Flurry lets you attack several more times at the expense
of many movement points. This is useful against creatures with bad
armor since you can do a great deal of damage quickly. Against
opponents with heavy armor and powerful magical defenses, combat hits
do almost nothing, so kicks are usually better.

Also, notice that there is no combat command (besides spells) that allows
you to hit every creature in the room at once. This is intentional,
but feel free to add this if you want.

There is also no ``martial arts'' or ``unarmed combat'' code just because
I want people to go out and get equipment. An appropriate place
to add this is marked near the end of the multi\_hit() function.
