\chapter{Crafting}

\section{Introduction}

There's crafting code in the game for objects, and I will leave that there,
but this is designed to supersede that code for mining and forging armor
and such. The idea is this. Every object is created by applying a process
to a certain number of arguments, or other objects. What happens is that
there's an area (106200 CRAFTGEN\_AREA\_VNUM) with a bunch of things
in it that generate things that can be used to craft other things. I hope
after this is done, I will be able to get back to the "society needs"
code and finish that correctly to give players lots of cool stuff that
they can find for mobs so they can get rewarded.


\section{Craftgen Objects}

These objects are in the craftgen area, and there will be special cases
later on, but these are the basics.

The objects get a name that's a single word or a string of words in quotes (so
that f\_word works correctly) and then they're set up as objects. If
you want them to be containers, make them not have the TH\_NO\_CONTAIN 
thing flag. 

Then they have a list of extra descriptions.

If they have an extra description called "names", that means that we will
make many versions of this object. For example, mineral chunks and
rock boulders and trees come in several types. That will let us make
objects of many different kinds. Use this when you want an object to
have many different forms. Only use this on base items, because the
code isn't general enough to handle it for crafted items. I may fix
this at some future date.

If they have an extra description called "gen", then this is setup data.
The gen edesc has a few words that give you information about the object:

\begin{itemize}

\item plural -- The object appears as some foo instead of a or an foo.

\item base -- This is a base item. Make the absolute utter raw materials base
  items. They are used for naming items.

\item noname -- Don't affix prefix names based on the materials being used to
  build this item. 

\end{itemize}

\section{Crafting Procedure}


Essentially, to craft an item, you need to give a command to carry out a
proces, and you need to have certain objects on or near you and maybe some
other requirements. The name of the crafting process is given in an edesc (so
you could have more than one crafting process to make an item if you so chose)
and it has a list of lines that have requirements for making the craft process
work.

The lines start with a command, and then some arguments telling what to do
with that command.

\begin{itemize}

\item wear, wield, hold -- These require someone to have certain objects
  equipped. You cannot specifiy a number here.

\item here -- This means there has to be an object on you or in the room
  you're in for the crafting to work.

\item use, made -- These specify objects that will disappear at the end of the
  crafting session. The difference is that the made command sets the prefix on
  the item name (like iron ingot vs ingot) and use just uses up the item.

\item room -- This word followed by some room sector flags are requirements
  for certain types of crafting (usually gathering).

\end{itemize}

You should be able to craft "iron ingot" and have it use up iron chunks
only. If you don't specify the specific item, it shouldn't work if the item
does accept a list of prefixes.


